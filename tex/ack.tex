%# -*- coding: utf-8-unix -*-
\begin{thanks}

四年的大学学习生活在即将划上一个句号,而于我的人生来说却仅仅只是一个逗号,我将面对新的征程的开始。本研究及论文是在我的导师蒋铃鸽教授和实验室的孙煜学长、梁轶学姐和张曼学姐的亲切关怀和耐心的指导下完成的。伟人、名人固然为我所崇拜,可是我更迫切地想要把我的敬意献给我的导师蒋铃鸽教授。也许我不是您最出色的学生,但您却是我所最尊敬的老师。您是如此的治学严谨,学识渊博,视野广阔,思想深刻,您用心为我营造一种良好的学术氛围,让我的论文更加的严谨。另外实验室的学长学姐们给了我很多的帮助,不论是对于论文内容、方向的把控,还是平日组会中的耐心解答,这都是我的论文得以完成的关键要素。同时,我还要感谢一下我的同窗们,如果没有你们的支持和倾心的协助,我是无法解决这些困难和疑惑,最终能够让本文顺利完成。

在本次论文设计过程中,蒋铃鸽老师对该论文从选题,构思到最后定稿的各个环节给予细心指引与教导,使我得以最终完成毕业论文设计。在学习中,老师严谨的治学态度、丰富渊博的知识、敏锐的学术思维、精益求精的工作态度以及侮人不倦的师者风范是我终生学习的楷模,导师们的高深精湛的造诣与严谨求实的治学精神,将永远激励着我。

上海交通大学,这里严谨的学风、优美的校园环境使我大学四年过的很充实和愉快。我的四年大学时光是在电子系度过的。在这四年时间里,我有幸和许多优秀的同学一齐学习,听睿智的电子系老师讲授课程。我学到了很多有用的知识,尤其是对我思想和方法上的指导。这些有用的东西一向对我大学的学习和生活有很重要的指导作用,我相信,这些东西将伴随我走完整个人生的道路。此刻回想起在电子系的日子,还是那么的温馨和惬意,我不能不感谢当时电子系的每一位同学和老师,跟你们在一齐学习、生活,那真是其乐融融,妙不可言!

感谢给我带给参考文献的学者们,多谢他们给我带给了超多的文献,使我在写论文的过程中有了参考的依据。感谢我的爸爸妈妈,感谢写作论文他们为我所付出的一切。养育之恩,无以回报,你们永远健康快乐是我最大的心愿。此时,我的情绪无法平静,从开始进入课题到论文的顺利完成,有多少可敬的师长、同学、朋友给了我无言的帮忙,是你们为我撑起一片天空,在那里请理解我诚挚的谢意。
\end{thanks}