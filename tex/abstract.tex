%# -*- coding: utf-8-unix -*-
%%==================================================
%% abstract.tex for SJTU Master Thesis
%%==================================================

\begin{abstract}
本文通过总结三个领域的最新进展:深度增强学习算法、数据中心网络中的数据流负载均衡算法和交换机处的缓冲区流量调度算法,针对5G Cloud RAN场景下的负载均衡问题提出了一个新的解决方案。此处的负载均衡是指广义的负载均衡,即包含了从端系统到交换机到端系统整体路径上的负载均衡,涵盖数据流的优先级标记、交换机缓存处的入队、出队操作及选路、拥塞信息采集等一系列过程。该解决方案是一个混合型的结构,对于短数据流采用性能极佳的DRILL分布式负载均衡算法,基于本地信息给出快速的均衡决策;利用离散形式的策略梯度算法和决定性(连续空间下)的actor-critic算法集中式地解决了长数据流的选路、流速率、优先级和对于网络架构中交换机处采用的MLFQ的队列阈值的实时部署,根据动态的网络流量情况,以及以往数据得到的经验,给出一种接近最优的流量优化方法。

\keywords{\large 负载均衡 \quad 增强学习 \quad 数据中心网络 \quad 5G Cloud RAN}
\end{abstract}

\begin{englishabstract}

This thesis deals with the general load balancing problem in data center networks. Under the scenario of the Cloud Radio Access Network of the fifth generation of wireless communication system, we propose a new paradigm for a combination of load balancing algorithm and flow scheduling algorithm. The system is a hybrid model with several hierarchies, namely centralized flow scheduling component and distributed load balancing component. The centralized flow scheduling scheme employs the state-of-the-art deep learning algorithm, Deep Deterministic Policy Gradient and Stochastic Policy Gradient algorithm, to solve the problem of assignment of rate limit, queue priorities and path selection for the long flows in the datacenter network and the thresholds of the Multi Level Feedback Queue commonly used in modern commodity switches. The distributed load balancing algorithm, which addresses load balancing problem of narrow sense, utilizes the recent model DRILL proposed by Soudeh Ghorbani, aiming at achieving lower flow complete time and lower total transmission time, not to mention the raising of rate of link utilization.

\englishkeywords{\large load balancing, reinforcement learning, data center network, 5G Cloud RAN}
\end{englishabstract}

